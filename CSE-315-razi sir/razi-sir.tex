\documentclass[12pt]{article}

\usepackage[utf8]{inputenc}
\usepackage{amsmath}
\usepackage{amssymb}
\usepackage{caption}
\usepackage{color}
\usepackage{float}
\usepackage{graphicx}
\usepackage{listings}
\usepackage{physics}
\usepackage{tikz}
\usepackage{listings}
\usepackage{subfiles}
\usepackage{enumerate}

\setlength{\parindent}{5em}
\setlength{\parskip}{1em}
\renewcommand{\baselinestretch}{1.5}

\title{	
	\textbf{CSE-315 Microprocessor}
	\endgraf\bigskip
}

\author{
	\Large{Waqar Hassan Khan}\\
	\Large{Student ID : 1505107}
}

\date{}

\begin{document}

\maketitle

\newpage
\tableofcontents
\newpage

%---------------------------------------------------------------------------------------
%lecture-1
\section{Chapter-9 : 8086/8088 specification}

\begin{itemize}
	\item virtually no differences between these two microprocessors. Both are packaged in 40-pin dual in-line package.
	
	\item\textbf{8086:}16 bit microprocessor with a 16bit data bus(A0-A15)\\
	\textbf{8088:}16 bit microprocessor with a 8bit data bus(A0-A7)\\
	major difference between 8086 and 8088.
	
	\item Minor differences\\
	\textbf{8086:}M/$\overline{IO}$;\textbf{8088:}IO/$\overline{M}$\\
	PIN34:-\textbf{8086:}$\overline{BHE}$/S7;\textbf{8088:}SS0\\
	
	\item power supply requirements:\\
	+5v with a supply voltage tolerance of $+-10\%$\\
	both 32F to 180F\\
	8086 and 8088 have 340 and 360 mA.
	
	\textbf{figures in the book}
	
	\subsection{Pin Connection}
	\begin{itemize}
		\item \textbf{AD7-AD0:}\\
		-8088 address/data bus lines\\
		-multiplexed address data bus\\
		-rightmost 8 bits of the memory address or I/O port whenever ALE=1 or ALE=0\\
		-high impedance state during hold acknowledge.
		
		\item \textbf{A15-A8:} \\
		-8088 address bus\\
		-high impedance state during hold acknowledge.
		
		\item \textbf{AD15-AD8:}\\
		-8086 address/data bus lines.\\
		-contains address bits when ALE=1\\
		-high impedance state during hold acknowledge.
		
		\item \textbf{A19/S6-A16/S3:}\\
		-multiplexed address/data bus lines\\
		-high impedance state during hold acknowledge.
		
		\item\textbf{S6:} always 0\\
		\item\textbf{S5:} indicated the condition of \textbf{\textcolor{red}{IF}} flag\\
		
		\begin{table}[H]
			\centering
			\begin{tabular}{|c|c|c|}
				\hline
				S4 & S3 & indicate segment accessed  during current bus cycle\\\hline
				
				0 & 0 & extra segment\\\hline
				0 & 1 & stack segment\\\hline
				1 & 0 & code or no segment\\\hline
				1 & 1 & data segment\\\hline
			\end{tabular}
		\end{table}
	
		\item -$\overline{RD}:$ if it is 0 then the data bus becomes receptive to data from memory or i/o devices connected to the system.\\
		-high impedance state during hold acknowledge.
		
		\item\textbf{READY:}\\
		-enters into wait state and remains idle if 0\\
		-no effect on operations of microprocessor if this pin is in logic state 1.
		
		\item\textbf{INTR:}\\
		-used to request a h/w interrupt.\\
		-if INTR=1 when IF=1 then microprocessor enters an interrupt acknowledge cycle after completion of current instruction
		
		\item\textbf{NMI:}\\
		-non maskable interrupt pin.\\
		-similar to INTR except do not check IF flag.
		
		\item$\overline{\textbf{TEST}}:$\\
		-an input that is tested by \textbf{wait} instruction.
		-if it is 0, the \textbf{WAIT} functions as \textbf{NOP}.\\
		-if 1 then \textbf{WAIT} waits for   $\overline{\textbf{TEST}}$ to become logic 0.
	\end{itemize}
\end{itemize}
%---------------------------------------------------------------------------------------

\newpage

%---------------------------------------------------------------------------------------
%lecture-2
\section{Chapter-9 : 8086/8088 specification}
\begin{itemize}
	\item\textbf{NMI:}\\
	-non-maskable interrupt pin \\
	-similar to INTR except that NMI does not check IF flag.
	
	\item\textbf{RESET:}\\
	-causes the microprocessor to reset if this pin remains high for a minimum of 4 clocking periods.\\
	-whenever the microprocessor  gets reset, it begins executing instructions at memory location FFFF0H and disables future interrupts by clearing IF.
	
	\item\textbf{CLK:}\\
	-provides the base timing signal to the microprocessor.\\
	-clock signal must have at least 33\% duty cycle(high for $\frac{1}{3}$ rd and low for $\frac{2}{3}$ of the period)
	
	\item\text{VCC}\\
	-power supply input\\
	-provides +5V
	
	\item\textbf{GND:}- 2 pins, both must be connected to ground.
	
	\item\textbf{MN/$\overline{MX}$}:-selects either minimum mode or maximum mode operations of microprocessor.
	
	\item\textbf{$\overline{BHE}$/S7}-both high enable.
	-use in 8086 to enable the most significant data bus bits(D15-D8) during a read or write\\
	-the state of S7 is always a logic1
	
	\subsection{Minimum Mode Pins}
	\begin{itemize}
		\item\textbf{IO/$\overline{M}$ or M/$\overline{IO}$}\\
		-selects memory or i/o 
		-indicates that microprocessors address\\ bus contains either a memory address or an i/o port address.\\
		-high impedance state during a hold acknowledge.
		
		\item \textbf{$\overline{WR}$:} \\
		-indicates that microprocessor is outputting data to a memory or io device.\\
		-data bus contains valid data for memory or io during the time, WR remains 0.
		
		\item \textbf{$\overline{INTA}$:} \\
		-a response to the INTR input pin.\\
		-used to gate the interrupt vector number onto the data bus in response to an interrupt request.
		
		\item \textbf{$\overline{ALE}$:} \\
		-address latch enable.\\
		-indicates that the microprocessor address/data bus contains address information.\\
		-the address can be a memory address or an i/o port.\\
		-does not float during a hold acknowledge.
		
		\item\textbf{DT/$\overline{R}$:} - data transmit or receive.\\
		-indicates that microprocessors data bus is transmitting(DT/$overline{R}$=1) or receiving(DT/$overline{R}$=(DT/$overline{R}$=0) data.\\
		-used to enable external data bus buffers.
		
		\item\textbf{DEN:}-data bus enable\\
		-activates external data bus buffers.
		
		\item\textbf{HOLD:} - requests a direct memory access (DMA).\\
		-if it is logic 1, microprocessor stops executing s/w and places its address, data and control bus at high impedance state.\\
		-if it is a logic 0, the microprocessor executes s/w normally.
		
		\item\textbf{HLDA:} - hold acknowledge\\
		-indicates that the microprocessor has entered the hold state.
		
		\item\textbf{$\overline{SS0}$} - equivalent to the S0 pin in the maximum mode operation.\\
		-it is combined with IO/$\overline{M}$ and DT/$\overline{R}$ to decode function of the current bus cycle.
		  
	\end{itemize}  

	\begin{table}[H]
		\centering
		\begin{tabular}{|c|c|c|c|}
			\hline
			IO/$\overline{M}$ & DT/$\overline{R}$ & $\overline{SS0}$ & \textbf{function}\\\hline
			
			0 & 0 & 0 & interrupt acknowledge\\\hline
			0 & 0 & 1 & memory card\\\hline
			0 & 1 & 1 & memory write\\\hline
			0 & 1 & 1 & halt\\\hline
			1 & 0 & 0 & opcode fetch\\\hline
			1 & 0 & 1 & I/O read\\\hline
			1 & 1 & 0 & I/O write\\\hline
			1 & 1 & 1 & passive/inactive\\\hline
		\end{tabular}
		\caption{bus cycle status(8088)[minimum mode]}
	\end{table}

		\begin{table}[H]
		\centering
		\begin{tabular}{|c|c|c|c|}
			\hline
			I$\overline{S2}$ & $\overline{S1}$ & $\overline{S0}$ & \textbf{function}\\\hline
			
			0 & 0 & 0 & interrupt acknowledge\\\hline
			0 & 0 & 1 & I/O card\\\hline
			0 & 1 & 1 & I/O write\\\hline
			0 & 1 & 1 & halt\\\hline
			1 & 0 & 0 & opcode fetch\\\hline
			1 & 0 & 1 & memory read\\\hline
			1 & 1 & 0 & memory write\\\hline
			1 & 1 & 1 & passive\\\hline
		\end{tabular}
		\caption{bus control functions generated by the bus controller 8088[maximum mode]}
	\end{table}
\end{itemize}
%---------------------------------------------------------------------------------------

\newpage

%---------------------------------------------------------------------------------------
%lecture-3
\section{8086/8088 hardware specifications}
\textbf{Maximum Mode Pins:} for using with external co-processors \\
\begin{itemize}
	\item$\overline{S2}, \overline{S1}, \overline{S0}$ - indicate the function of current bus cycle.\\
	-normally decoded 8288 bus controller.
	
	\item $\overline{R1}/\overline{GT1} and \overline{R0}/\overline{GT0}$ - request/grant pins \\
	-requests direct memory access(DMA)\\
	-used to both request and grant DMA operation.
	
	\item\textbf{$\overline{LOCK}$} - used to lock peripheral off the system.
	
	\item \textbf{OS1 and OS2} - queue status bit.\\
	-show status of the internal instruction queue.\\
	-accessed by numeric co-processor(8087)
	
	\begin{table}[H]
		\centering
		\begin{tabular}{|c|c|c|}
			\hline
			QS1 & QSA & Function\\\hline
			0 & 0 & queue is idle\\\hline
			0 & 1 & first byte of opcode\\\hline
			1 & 0 & queue is empty\\\hline
			1 & 1 & subsequent byte of opcode\\\hline
		\end{tabular}
	\end{table} 
\end{itemize}

\subsection{clock generator-8284A}
\begin{figure}[H]
	\centering
	\captionsetup{justification=centering}
	\includegraphics[width = 0.5\textwidth]{image/8284A.png}
	\caption{
		8284A clock generator
	}
	\label{fig:fib}
	
\end{figure}

\begin{itemize}
	\item \textbf{Basic functions:}\\
	-clock generation\\
	-RESET synchronization\\
	-READY synchronization\\
	-TTL-level peripheral clk signals
	
	\item\textbf{pin functions}
	
	\item\textbf{AEN1 and AEN2 (address enable)} - qualify the bus ready signals RDY1 and RDY2 respectively.\\
	-wait states are generated by the READY pin of microprocessor, which is controlled by $\overline{AEN1}$ and $\overline{AEN2}$
	
	\item\textbf{RDY1 and RDY2} - Bus ready inputs.\\
	-cause wait states in conjunction with $\overline{AEN1}$ and $\overline{AEN2}$ pins.
	
	\item$\overline{\textbf{ASYNC}}:$ - READY synchronization.\\
	-selects either one or two stages of synchronization for RDY1 and RDY2 inputs.
	
	\item\textbf{READY} - an output pin that connects to microprocessors READY input.\\
	-synchronized with RDY1 and RDY2 inputs.
	
	\item \textbf{X1 and X2:} - crystal oscillator pins.\\
	-connect to an external crystal which is used as the timing source for the clock generator and all its functions.
	
	\item $\textbf{F/}\overline{\textbf{C:}}$ - Frequency/crystal select input.\\
	- chooses the clocking source.\\
	- if it is held high, an external clock is provided to the EFI pin.\\
	- if it is held low, the internal crystal oscillator provides the timing signal.
	
	\item \textbf{EFI:} - External Frequency Input.\\
	- supplies timing whenever F/$\overline{C}$ is held high.
	
	\item\textbf{CLK:} - clock output pin, which provides clock input to microprocessor and other components.\\
	
	- output signal is $\frac{1}{3}$ of crystal or EFI input freq. and has a duty cycle of 33\%(as required by 8086/8088).
	
	\item \textbf{PCLK:} - peripheral clock.\\
	-$\frac{1}{6}$ of the crystal or EFI input freq. and has a 50\% duty cycle.
	
	\item \textbf{OSC:} - oscillator output.\\
	- at same freq. as the crystal or EFI input.\\
	- provides an EFI input to other 8284A in a multi-processor system.
	
	\item $\overline{\textbf{RES}}:$ - reset input.\\
	- often connected to an RC network that provides power on resetting.
	
	\item\textbf{RESET:} - reset output.\\
	- connected to microprocessors RESET input pin.
	
	\item \textbf{CSYNC:} - clock synchronization.\\
	- used whenever the EFI input provides synchronization in a multi-processor system.\\
	-if the internal oscillator is used, this pin must be grounded. 
	  
\end{itemize}
%---------------------------------------------------------------------------------------


%---------------------------------------------------------------------------------------
%lecture-4
\section{Chapter-9 : 8086/8088 specification}
\subsection{internal block diagram of 8284A clock generator}
\begin{figure}[H]
	\centering
	\captionsetup{justification=centering}
	\includegraphics[width = 1.0\textwidth]{image/8284AINT.png}
	\caption{
		internal block diagram of 8284A clock generator
	}
	
\end{figure}

\begin{itemize}
	\item if a crystal is attached to X1 and X2, the oscillator generates a square wave signal at the same frequency of the crystal.\\
	\item $CLK=\frac{frequency}{3}; PCLK=\frac{frequency}{6}$\\
\end{itemize}

\subsection{operation of the RESET section}
\begin{figure}[H]
	\centering
	\captionsetup{justification=centering}
	\includegraphics[width = 0.5\textwidth]{image/reset_8284A.png}
	\caption{
		RESET operation of 8284A clock generator
	}
	
\end{figure}
\begin{itemize}
	\item $\overline{\textbf{RES}}$ goes through a \textbf{Schmitt Trigger} and a \textbf{D-type} flip-flop, which ensures meeting timing requirements of microprocessors RESET input.
	
	\item the circuit applies RESET to microprocessor at a negative edge(1->0) and the microprocessor samples the RESET the signal at the positive edge(0->1).
	
	\item when power is first applied to the system, the RC circuit provides a logic 0 to $\overline{\textbf{RES}}$
	
	\item after a short time $\overline{\textbf{RES}}$ becomes logic 1, as the capacitor changes to +5v through the resistor.
	
	\item a push button allows the microprocessor to be reset by an operator.
	
	\item correct RESET timing requires the RESET input to become a logic 1 no later than 4 clock cycles after the power is applied, and held high for at least 50 micro-seconds.
	
	\item RESET goes high in 4 clock by FF. RESET stays high for 50 micro-seconds by RC.
	
\end{itemize}

\subsection{Bus buffering and latching}
\begin{itemize}
	\item the address/data bus on the microprocessor is multiplexed(shared) to reduce the \# of pins, which, on the other hand, burdens with the task of extracting or de-multiplexing info from these pins.
	
	\item Why not leave the buses multiplexed??\\
	memory and i/o require that the address remains valid and stable throughout a read or write cycle. If the buses are multiplexed , the address can get changed causing read or write in wrong locations.
	
	\item All computer systems have 3 types of buses:
	\begin{enumerate}
		\item Address bus: provides memory address or i/o port of \#s 
		\item Data bus: transfers data between microprocessor and memory and i/o.
		\item control bus: provides control signals to memory and i/o.
	\end{enumerate}
	
\end{itemize}

\textbf{Basic of de-multiplexing- images in the lecture}
%---------------------------------------------------------------------------------------

\newpage

%---------------------------------------------------------------------------------------
%lecture-5
\section{Demultiplexing 8088}
\textbf{9th image, figure would added later}\\
\begin{itemize}
	\item 74LS373 latches(OE for output enable and G or LE for latch enable inside) are used to demultiplex address data bus connections and address/status bus connections.
	
	\item 74LS373 passes inputs to outputs like wires when ALE is logic 1; when ALE returns to logic 0, the latches remember the inputs at the time of the change to logic 0. 
\end{itemize}

\subsection{Demultiplexing 8086}
\textbf{9th image, figure would added later}\\
\begin{itemize}
	\item Difference from 8088, AD15-AD8 and $\overline{BHE}$/S7
	\item $\overline{BHE}$ selects a high order memory bank in a 16-bit memory system in 8086.
\end{itemize}

\subsection{Buffered System}
\begin{itemize}
	\item If more than 10 unit loads are attached to any bus pins, the entire microprocessor system must be buffered(Buffer provides amplification in a digital circuit to drive output loads enabling more TTL unit loads to be driven.)
	
	\item The demultiplexed pins are already buffered by the 74LS373 latches.
	
	\item A fully buffered signal will introduce a timing delay which causes no difficulty unless memory or i/o devices are used that function at near the maximum speed of the bus.
\end{itemize}

\subsection{Fully buffered 8088}
\textbf{image from 10th image}

\begin{table}[H]
	\centering
	\begin{tabular}{|c|c|c|c|}
		\hline
		microprocessor & 74LS244(octal buffer) & 74LS245(octal bidirectional bus buffer) & 74LS373\\\hline
		8088 & 2 & 1 & 2\\\hline
		8086 & 1 & 2 & 3\\\hline
	\end{tabular}
\end{table}
%---------------------------------------------------------------------------------------

\newpage

%---------------------------------------------------------------------------------------
%lecture-6
\section{Memory Interfaces}
\begin{itemize}
	\item 4 common types of memory: ROM, flash memory(EEPROM), static random access memory(SRAM) and dynamic random access memory(DRAM).
	
	\item pin connections common to all memory devices: address inputs, data inputs(inputs/outputs), some type of selection input and at least one control input used to select a read or write operation.
	
	\item control connections:\\
	ROM-only one control input($\overline{OE}$ or $\overline{G}$)-output enable and gate.\\
	RAM-one $R/\overline{W}$ or 2 ($\overline{WE}/\overline{W} and \overline{OE}/\overline{G}$)-control input do not get activated at the same time.
\end{itemize}

\subsection{ROM memory(nonvolatile memory)}
\begin{itemize}
	\item permanently stores programs that are resident to the system and must not change when power supply is disconnected(permanently programmed)
	
	\item EPROM(erasable programmable rom): programmed using a device called EPROM programmer; erasable if exposed to high-intensity uv light for about 20 minutes or less.
	
	\item PROM(programmable ROM): programmed by burning open tiny Ni-Chrome or Silicon Oxide fuses. Once programmed, it can't be erased
	
	\item Read Mostly Ram(RMM) or flash memory or EEPROM(electrically erasable programmable rom) or EAROM(electrically altered rom) or NOVRAM(non-volatile ram). Electrically erasable, however, needs more time to erase than a normal RAM.
	
	\item delays in operation of an EPROM:\\
	$t_{acc}1$: address to output delay.\\
	$t_{OH}$: address to output hold.\\
	$t_{co}$: chip select to output delay.\\
	$t_{OF}$: chip deselect to output float.
	
	\textbf{timing digram of eprom- page12} 
\end{itemize}

\subsection{Static Memory or Static RAM(SRAM) or Volatile memory}

\begin{itemize}
	\item Retain data as long as DC power is applied (no data without power)
	
	\item Difference between ROM and RAM:\\
	RAM->written under normal operation;
	ROM->programmable outside the computer and is normally read.
\end{itemize}

\subsection{DRAM}
\begin{itemize}
	\item DRAM is essentially same as SRAM, except that it retains data for only 2 or 4ms on an integrated capacitor.
	
	\item Alter 2 or 4ms, the contents of DRAM must be completely rewritten(refreshed) because the capacitors(which store logic 1/0) lose their charges.
	
	\item Refreshing also occurs during a write, a read or during a special refresh cycle.
\end{itemize}
%---------------------------------------------------------------------------------------

\newpage

%---------------------------------------------------------------------------------------
%lecture-7
\section{Lecture 7}

\subsection{Dynamic RAM (DRAM)}
\begin{itemize}
	\item Have much larger sizes compared to SRAM. Its obvious disadvantage is requirement of many address pins. To reduce the number of address pins, the address inputs are multiplexed.
	
	\item Example of multiplexed address pins: 64k X 4 DRAM
	
	\textbf{image from image13}\\
	
	A0-A7 : Address inputs\\
	$\overline{CAS}$ : Column Address Strobe\\
	DQ1-DQ4 : Data-in/Data-out\\
	$\overline{G}$ : Output enable\\
	$\overline{RAS}$ : Row Address Strobe\\
	VDD : +5v supply\\
	VSS : Ground\\
	$\overline{W}$ : Write enable
	
	\item Here for 64k, 16 address pins are required. However, 8 address pins are used through multiplexed\\
	 
	- At first, A0-A7 are stored in internal row latch as row address through  enabling $\overline{RAS}$\\
	- Next A8-A15 are stored in internal column latch as column address through enabling $\overline{CAS}$
	
	\item Address multiplexer of 65k X 4\\
	\textbf{image from image13}\\
	
	\item If $\overline{RAS}$ is 0, then 'A' pins get connected and A0-A7 are stored in the \textbf{Internal Row Address} latch
	
	\item The Internal Row Address latch is edge triggered and so the row address get captured into the latch before the address changes to column address.
	
	\item If $\overline{RAS}$ is 1, then the 8-pins get connected and A8-A15 are stored in Internal Column Address Latch  
\end{itemize}

\subsection{Address Decoding}
\begin{itemize}
	\item Decoding address sent from microprocessor is necessary, as without it, only one memory device can be connected to a microprocessor.
	
	\item Another reason of decoding is possible mismatch(or difference) between the \# of connections in microprocessor and memory\\.
	e.g. a 2k X 8 EPROM(2716) has 11 address pins, whereas the microprocessor 8088 has 20 address pins.
	
	\item Simple NAND Gate Decoder:\\
	\textbf{image from image14}
	
	\item Here, the 2k EPROM is decoded at memory address locations \textbf{FF800H-FFFFFH} (the FF8 corresponds to 1's in left 9 positions)
	
	\item In such a decoding, one NAND gate decoder selects 1 2k EPROM out of many 2k EPROM that appear in 1M address space.
	
	\item The obvious disadvantage is that each EPROM needs one NAND gate decoder.
	
	\item 3-8 line decoder(74LS138)
	
	\textbf{image from image14}
	
	\item Input : enable=001 and selection(ABC)=n\\
	Output : nth output=0; rest are 1
	
	\item Input : enable!=001\\
	Output : all outputs are 1
	
	\item A16-A19 must have 1 for activating 74138, therefore, all addresses begin with "1111" at the left
	
	\item 8 2764(EPROM) (each 8k X 8, having 13 pins) are decoded over address space F0000H-FFFFFH(64k X 8) 
	
\end{itemize}
%---------------------------------------------------------------------------------------

\end{document}


