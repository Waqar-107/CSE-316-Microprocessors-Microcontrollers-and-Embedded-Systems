\documentclass[12pt]{article}

\usepackage[utf8]{inputenc}
\usepackage{amsmath}
\usepackage{amssymb}
\usepackage{caption}
\usepackage{color}
\usepackage{float}
\usepackage{graphicx}
\usepackage{listings}
\usepackage{physics}
\usepackage{tikz}
\usepackage{listings}
\usepackage{subfiles}
\usepackage{enumerate}

\setlength{\parindent}{5em}
\setlength{\parskip}{1em}
\renewcommand{\baselinestretch}{1.5}

\title{	
	\textbf{CSE-315 Microprocessor}
	\endgraf\bigskip
}

\author{
	\Large{Waqar Hassan Khan}\\
	\Large{Student ID : 1505107}
}

\date{}

\begin{document}

\maketitle

\section{Chapter-9 : 8086/8088 specification}

\begin{itemize}
	\item virtually no differences between these two microprocessors. Both are packaged in 40-pin dual in-line package.
	
	\item\textbf{8086:}16 bit microprocessor with a 16bit data bus(A0-A15)\\
	\textbf{8088:}16 bit microprocessor with a 8bit data bus(A0-A7)\\
	major difference between 8086 and 8088.
	
	\item Minor differences\\
	\textbf{8086:}M/$\overline{IO}$;\textbf{8088:}IO/$\overline{M}$\\
	PIN34:-\textbf{8086:}$\overline{BHE}$/S7;\textbf{8088:}SS0\\
	
	\item power supply requirements:\\
	+5v with a supply voltage tolerance of $+-10\%$\\
	both 32F to 180F\\
	8086 and 8088 have 340 and 360 mA.
	
	\textbf{figures in the book}
	
	\subsection{Pin Connection}
	\begin{itemize}
		\item \textbf{AD7-AD0:}\\
		-8088 address/data bus lines\\
		-multiplexed address data bus\\
		-rightmost 8 bits of the memory address or I/O port whenever ALE=1 or ALE=0\\
		-high impedance state during hold acknowledge.\\
		
		\item \textbf{A15-A8:} \\
		-8088 address bus\\
		-high impedance state during hold acknowledge.\\
		
		\item \textbf{AD15-AD8:}\\
		-8086 address/data bus lines.\\
		-contains address bits when ALE=1\\
		-high impedance state during hold acknowledge.\\
		
		\item \textbf{A19/S6-A16/S3:}\\
		-multiplexed address/data bus lines\\
		-high impedance state during hold acknowledge.\\
		
		\item\textbf{S6:} always 0\\
		\item\textbf{S5:} indicated the condition of \textbf{\textcolor{red}{IF}} flag\\
		
		\begin{table}[H]
			\begin{tabular}{|c|c|c|}
				\hline
				S4 & S3 & indicate segment accessed  during current bus cycle\\\hline
				
				0 & 0 & extra segment\\\hline
				0 & 1 & stack segment\\\hline
				1 & 0 & code or no segment\\\hline
				1 & 1 & data segment\\\hline
			\end{tabular}
		\end{table}
	
		\item -$\overline{RD}:$ if it is 0 then the data bus becomes receptive to data from memory or i/o devices connected to the system.\\
		-high impedance state during hold acknowledge.\\
		
		\item\textbf{READY:}\\
		-enters into wait state and remains idle if 0\\
		-no effect on operations of microprocessor if this pin is in logic state 1.\\
		
		\item\textbf{INTR:}\\
		-used to request a h/w interrupt.\\
		-if INTR=1 when IF=1 then microprocessor enters an interrupt acknowledge cycle after completion of current instruction\\
		
		\item\textbf{NMI:}\\
		-non maskable interrupt pin.\\
		-similar to INTR except do not check IF flag.\\
		
		\item$\overline{\textbf{TEST}}:$\\
		-an input that is tested by \textbf{wait} instruction.
		-if it is 0, the \textbf{WAIT} functions as \textbf{NOP}.\\
		-if 1 then \textbf{WAIT} waits for   $\overline{\textbf{TEST}}$ to become logic 0.\\
	\end{itemize}
\end{itemize}


\end{document}


